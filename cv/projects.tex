
\header{\textbf{Projects}}
\vspace{1pt}
{\textbf{WatchDog}} {\sl Apache Kafka,ElasticSearch,Kibana,Bash,Golang} \hfill \textbf{\href{https://github.com/zakhaev26/WatchDog}{\textbf{\textcolor{linkcolor}{Source}}}} \\[1.7pt]

A \textbf{distributed fault tolerant Server Monitoring Tool} written in \textbf{Golang} for \textbf{maintaining vigilance and oversight over servers providing sysadmins with near real-time insights into server health and performance.}\\
\vspace{-0.5em}
\begin{itemize}
    \itemsep-0.2em
    \item Engineered a \textbf{fault-tolerant microservice} architecture for continuous \textbf{CPU health logging}, using \textbf{Apache Kafka-Zookeeper} for distributed monitoring.
    \item Implemented \textbf{automated alerts} via \textbf{SMTP} mails when predefined avg. CPU load thresholds are exceeded.
    \item Used \textbf{Protocol Buffers} for communication between Microservices, resulting in \textbf{55.769 \% decrease} in network payload sizes per Publish/Subscribe Calls in comparison to traditional REST Calls. 
    \item Designed a user-friendly \textbf{CLI tool} and \textbf{bash scripts} using \textbf{Golang \& open-source frameworks} for easy setup.Also \textbf{Tested} responses to \textbf{CPU abnormalities} and \textbf{stress testing with \~10,000 cURL requests to Elastic Cloud with 0 Failures.}
\end{itemize}
\vspace*{-3pt}

{\textbf{Tunescape}} {\sl Next.js,Node.js,Express.js,MongoDB,AWS,} \hfill \textbf{\href{https://github.com/zakhaev26/Tunescape}{\textbf{\textcolor{linkcolor}{Source}}}} | \textbf{\href{https://github.com/zakhaev26/WatchDog}{\textbf{\textcolor{linkcolor}{Link}}}} \\[1.7pt]

A \textbf{Cloud Based} Music Streaming Web Application that allows anyone to push their favourite songs in \textbf{Public cloud} and and serve it to people in low latency.\\
\vspace{-0.3em}
\begin{itemize}
    \itemsep-0.2em
    \item \textbf{Refactored} the \textbf{Entire Client Code} from \textbf{EJS to Next.js} for better \textbf{compatibility with server side APIs} and used \textbf{Firebase} for Authentication.
    \item Wrote \textbf{server-side logics} to store raw music files efficiently in an \textbf{AWS S3 Bucket}, with corresponding references saved in \textbf{MongoDB}
    \item Utilized \textbf{AWS Cloudfront} as a \textbf{Content Delivery Network (CDN)} to ensure \textbf{low-latency delivery} of music files on user requests.
\end{itemize}
\vspace*{-1pt}

{\textbf{KoelCLI}} {\sl C,PortAudio,Javascript,Node.js,Linux,Bash} \hfill \textbf{\href{https://github.com/zakhaev26/FeatherCLI}{\textbf{\textcolor{linkcolor}{Source}}}}

A \textbf{Command Line Interface (CLI) Tool} to play Music using \textbf{Terminal} in \textbf{Debian OS}.
\vspace{-0.6em}
\begin{itemize}
    \itemsep-0.2em
    \item Implemented \textbf{FFMPEG} and \textbf{Bash Scripts} which \textbf{decreased network consumption by 40-60\%} \& \textbf{enhance sound quality, reduce lossy compressions} by converting \textbf{WAV file} from the fetched \textbf{MP3} using \textbf{YT-dLP} in the \textbf{Host's System}.
    \item Wrote an \textbf{Audio Playback Engine, entirely in C} that \textbf{consumes WAV files and plays low-latency Audio} using \textbf{PortAudio Library}. 
\end{itemize}
\vspace*{2mm}
